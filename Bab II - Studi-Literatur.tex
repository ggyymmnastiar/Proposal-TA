\chapter{TINJAUAN PUSTAKA}

\section{Tinjauan Teoritis}

\subsection{Usaha Mikro, Kecil, dan Menengah (UMKM)}

Usaha Mikro, Kecil, dan Menengah (UMKM) merupakan sektor ekonomi yang sangat strategis di Indonesia, karena berkontribusi besar terhadap perekonomian nasional dan penyerapan tenaga kerja. Menurut Kementerian Koperasi dan UKM bekerja sama dengan Badan Pusat Statistik (BPS), pendataan lengkap UMKM pada tahun 2023 menunjukkan bahwa terdapat kolaborasi untuk membangun basis data tunggal usaha mikro hingga menengah di seluruh provinsi \cite{kemenkopukm_bps2023}. Sektor UMKM juga penting dalam konteks pemulihan ekonomi dan pemerataan, mengingat mayoritas usaha di Indonesia termasuk dalam kategori mikro atau kecil.

Dalam era digital, UMKM menghadapi tantangan sekaligus mendapat peluang besar. Bappenas telah mendorong transformasi digital bagi UMKM sebagai bagian dari strategi pembangunan nasional, menekankan bahwa digitalisasi tidak hanya memperluas pasar tetapi juga meningkatkan efisiensi dan kapasitas inovasi para pelaku UMKM \cite{bappenas2020_umkmdigital}. Adopsi teknologi digital menjadi faktor kunci agar UMKM dapat tetap relevan dan kompetitif di pasar modern.

Selain itu, penetrasi internet yang semakin tinggi di Indonesia juga membuka potensi signifikan bagi UMKM. Berdasarkan survei APJII tahun 2024, tingkat penetrasi internet di Indonesia mencapai 79,5\%, dengan lebih dari 221 juta pengguna aktif. Angka ini menunjukkan bahwa akses digital semakin merata, memberikan landasan bagi UMKM untuk memanfaatkan kanal online — baik untuk pemasaran, interaksi dengan pelanggan, maupun pengumpulan umpan balik digital.


\subsection{Transformasi Digital pada UMKM}
% Menjelaskan konsep digitalisasi, e-commerce, dan dampak teknologi terhadap daya saing UMKM.

\subsection{Natural Language Processing (NLP)}
% Menguraikan pengertian NLP, komponen utamanya, serta penerapan NLP dalam analisis teks berbahasa Indonesia.

\subsection{Analisis Sentimen}
% Menjelaskan konsep dasar analisis sentimen, pendekatan yang digunakan (lexicon-based, machine learning, deep learning), serta aplikasinya dalam bisnis dan UMKM.

\subsection{Klasifikasi Teks}
% Membahas pengertian klasifikasi teks, metode umum seperti Naive Bayes, SVM, BERT, serta metrik evaluasi model seperti accuracy, precision, recall, dan F1-score.

\section{Penelitian Terdahulu}
% Menguraikan ringkasan penelitian sebelumnya yang relevan dengan topik NLP, analisis sentimen, dan UMKM.
