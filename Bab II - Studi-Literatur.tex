\chapter{TINJAUAN PUSTAKA}

\section{Tinjauan Teoritis}

\subsection{Usaha Mikro, Kecil, dan Menengah (UMKM)}

Usaha Mikro, Kecil, dan Menengah (UMKM) merupakan sektor yang memiliki peran strategis dalam mendukung perekonomian nasional. Berdasarkan Undang-Undang Nomor 20 Tahun 2008, UMKM didefinisikan sebagai usaha produktif milik perorangan dan/atau badan usaha perorangan yang memenuhi kriteria tertentu terkait kekayaan bersih atau hasil penjualan tahunan. UMKM mencakup tiga kategori utama, yaitu usaha mikro, usaha kecil, dan usaha menengah, yang dibedakan berdasarkan jumlah aset dan omzet tahunan.

Menurut data Kementerian Koperasi dan Usaha Kecil dan Menengah (Kemenkop UKM), sektor UMKM berkontribusi sebesar 61\% terhadap Produk Domestik Bruto (PDB) Indonesia serta menyerap lebih dari 97\% tenaga kerja nasional \cite{kemenkopukm2024_umkmdata}. Jumlah pelaku UMKM di Indonesia mencapai lebih dari 65 juta unit pada tahun 2024, menunjukkan peran penting sektor ini dalam memperkuat struktur ekonomi domestik dan mengurangi tingkat pengangguran \cite{jokowi_umkm_pdb2024}.

Selain kontribusinya terhadap perekonomian, UMKM juga memiliki fungsi sosial, seperti menciptakan pemerataan ekonomi dan mendukung pengembangan wilayah. Namun, sektor UMKM masih menghadapi sejumlah tantangan, antara lain keterbatasan akses pembiayaan, rendahnya literasi digital, dan kurangnya pemanfaatan teknologi dalam proses bisnis \cite{bappenas2023_umkmtransformasi}.

Dalam konteks ekonomi digital, penguatan daya saing UMKM menjadi kunci utama agar dapat bertahan di tengah disrupsi teknologi. Digitalisasi memberikan peluang bagi UMKM untuk memperluas pasar, meningkatkan efisiensi operasional, serta memahami kebutuhan pelanggan melalui pemanfaatan data digital. Oleh karena itu, integrasi teknologi seperti \textit{Natural Language Processing} (NLP) dalam analisis umpan balik pelanggan dapat menjadi langkah strategis dalam mendukung pengembangan UMKM yang adaptif dan berkelanjutan.


\subsection{Transformasi Digital pada UMKM}
% Menjelaskan konsep digitalisasi, e-commerce, dan dampak teknologi terhadap daya saing UMKM.

\subsection{Natural Language Processing (NLP)}
% Menguraikan pengertian NLP, komponen utamanya, serta penerapan NLP dalam analisis teks berbahasa Indonesia.

\subsection{Analisis Sentimen}
% Menjelaskan konsep dasar analisis sentimen, pendekatan yang digunakan (lexicon-based, machine learning, deep learning), serta aplikasinya dalam bisnis dan UMKM.

\subsection{Klasifikasi Teks}
% Membahas pengertian klasifikasi teks, metode umum seperti Naive Bayes, SVM, BERT, serta metrik evaluasi model seperti accuracy, precision, recall, dan F1-score.

\section{Penelitian Terdahulu}
% Menguraikan ringkasan penelitian sebelumnya yang relevan dengan topik NLP, analisis sentimen, dan UMKM.
