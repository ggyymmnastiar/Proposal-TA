\chapter{PENDAHULUAN}

\section{Latar Belakang}

Usaha Mikro, Kecil, dan Menengah (UMKM) memainkan peran sangat penting dalam perekonomian Indonesia. Menurut pernyataan Presiden Joko Widodo pada tahun 2024, terdapat sekitar 65 juta unit UMKM yang menyumbang sekitar 61 \% dari Produk Domestik Bruto (PDB) nasional serta menyerap sekitar 97 \% tenaga kerja. \cite{jokowi_umkm_pdb2024} Kondisi ini menjadikan UMKM sebagai pilar utama dalam menciptakan lapangan kerja dan mendukung stabilitas sosial-ekonomi.

Dalam era transformasi digital yang pesat, peluang bagi UMKM untuk memperluas jangkauan melalui platform daring semakin terbuka lebar. Data dari survei Asosiasi Penyelenggara Jasa Internet Indonesia (APJII) menunjukkan bahwa penetrasi internet di Indonesia mencapai 79,5 \% pada tahun 2024, dengan jumlah pengguna internet mencapai sekitar 221,6 juta jiwa. \cite{apjii2024} Tingginya penetrasi internet ini membuka peluang besar bagi UMKM untuk memanfaatkan marketplace, media sosial, dan kanal digital lainnya sebagai media interaksi dan pemasaran kepada pelanggan.

Salah satu aset data penting yang dapat digarap oleh UMKM adalah umpan balik pelanggan dalam bentuk ulasan atau komentar di platform digital. Ulasan tersebut menyimpan informasi strategis mengenai perasaan konsumen (sentimen), aspek produk yang dihargai, atau elemen layanan yang perlu diperbaiki. Namun, volume umpan balik dalam bentuk teks yang sangat besar serta sifatnya yang tidak terstruktur menjadi hambatan bagi analisis manual. Banyak pelaku UMKM belum memiliki kapabilitas teknis atau sumber daya untuk mengeksplorasi data tersebut secara sistematis, sehingga potensi insight dari konsumen belum dimanfaatkan secara optimal.

Teknologi \emph{Natural Language Processing} (NLP) memberikan solusi untuk tantangan tersebut. NLP memungkinkan analisis otomatis terhadap teks, sehingga opini pelanggan dapat diklasifikasikan menurut polaritas (positif, negatif, netral) dan dikategorikan menurut tema atau aspek tertentu, seperti kualitas produk, harga, layanan pelanggan, dan pengiriman. Dengan cara ini, UMKM dapat memperoleh wawasan berbasis data untuk memperbaiki produk, meningkatkan layanan, dan membuat keputusan strategis yang lebih tepat.

Beberapa penelitian telah mengeksplorasi analisis sentimen dalam konteks UMKM. Permana, Noviyanto, dan Kristiyanti (2023) melakukan sentiment analysis terhadap opini masyarakat mengenai UMKM pada media sosial Twitter menggunakan metode Naïve Bayes. \cite{permana2023_umkm_twitter} Penelitian ini memberikan bukti bahwa metode klasik sederhana dapat digunakan untuk memahami sentimen publik terhadap UMKM, namun penelitian tersebut terbatas pada satu platform dan volume data relatif kecil.

Keterbatasan penelitian sebelumnya menegaskan perlunya penelitian yang lebih komprehensif yang tidak hanya melakukan analisis sentimen tetapi juga klasifikasi aspek dari umpan balik pengguna. Selain itu, penelitian lanjutan harus mempertimbangkan karakteristik bahasa Indonesia agar model NLP lebih relevan dan akurat dalam konteks lokal.

Oleh karena itu, penelitian ini bertujuan untuk menganalisis sentimen dan mengklasifikasikan umpan balik pengguna terhadap produk UMKM dengan bantuan teknik NLP. Secara teoritis, penelitian ini diharapkan berkontribusi terhadap pengembangan metodologi NLP untuk teks berbahasa Indonesia di domain UMKM. Secara praktis, hasil penelitian ini dapat membantu pelaku UMKM memahami persepsi pelanggan, meningkatkan kualitas produk dan layanan, serta merumuskan strategi pemasaran berbasis data yang lebih efektif.


